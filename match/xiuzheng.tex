% fleqn,控制公式左对齐
\documentclass[fleqn]{article}
\usepackage{xeCJK}
\usepackage{listings}

% mathtools数学环境
\usepackage{mathtools}
% 控制页边距
\usepackage[a4paper,left=3cm,right=3cm,top=2cm,bottom=3cm]{geometry}
\setCJKmainfont{SimSun}

% title
\title{重难点演练:有理数的四则运算2}
% 不显示时间
%\date{}
% 以上基本不用修改

% 自定义页眉
\usepackage{fancyhdr}
\pagestyle{fancy}
\chead{找到错误的答案,拿老师手里的卡}
\cfoot{学而思暑期班\phantom{how}第3讲随材}
\begin{document}
  % 这里设置标题的样式
  \hspace{6cm}{\Large 错误答案修正}
  
	\noindent 老师给的答案就一定是正确的吗?答案是不一定的!有时候可能你是对的,如果你坚信老师的答案错了,那么欢迎告诉老师!
	如果你错了,老师会给你讲解明白;如果老师错了,会有卡片等着大家哦!
  \[\text{1、}1.25\times\frac59\times(-4)\times0.6 \text{答案应为}-\frac53 \text{(重难点演练:有理数的四则运算1乘除法),发现人:}\]
  \[\text{5、} (+1\frac45)-(+\frac23)-(-\frac15)-(+\frac13)\text{答案应为}1\text{(重难点演练:有理数的四则运算2),发现人:季欣然,宋子越}\]
  \[\text{5、}-3^2-(-3)^2\times(-\frac13)+(-3)^3\div3\text{答案应为}-15\text{(重难点演练3:有理数的混合运算),发现人:宋子越}\]
\end{document}
