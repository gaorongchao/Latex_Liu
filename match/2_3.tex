% fleqn,控制公式左对齐
\documentclass[fleqn]{article}
\usepackage{xeCJK}

% mathtools数学环境
\usepackage{mathtools}
% 控制页边距
\usepackage[a4paper,left=2cm,right=3cm,top=2cm,bottom=3cm]{geometry}
\setCJKmainfont{SimSun}

% title
\title{重难点演练:有理数的四则运算2}
% 不显示时间
%\date{}
% 以上基本不用修改

% 自定义页眉
\usepackage{fancyhdr}
\pagestyle{fancy}
\chead{减去一个数等于加上这个数的相反数}
\cfoot{学而思暑期班\phantom{how}第二讲随材}
\begin{document}
  % 这里设置标题的样式
  \hspace{4cm}{\Large 重难点演练:有理数的四则运算2}
  \\
  \\
  
  \noindent有理数的加减法:\\
  \[\text{1、} -5\frac56+(-9\frac23)+17\frac34+(-3\frac12)\]
  
  \[\text{2、} (-3\frac23)+(-2.4)-(-\frac13)-(-4\frac25)\]
  
  \[\text{3、} -12-(-25)+(-32)-(+4)+10\]
  
  \[\text{4、} (+2\frac34)-(-1\frac12)+(-\frac56)-(-\frac38)-(+4\frac23)\]
  
  \[\text{5、} (+1\frac45)-(+\frac23)-(-\frac15)-(+\frac13)\]
  有理数乘除法:
  \[\text{6、} (-0.25)\times0.5\times(-4\frac27)\times4\]
  
  \[\text{7、} (\frac79-\frac56+\frac34-\frac7{18})\times(-36)\]
  
  \[\text{8、} 3\frac12\times(-\frac57)-(-\frac57)\times2\frac12-\frac57\times(-\frac12)\]
  
  \[\text{9、} 91\frac{71}{72}\times(-36)\]
  
  \[\text{10、} -72\times2\frac14\times\frac49\div(-3\frac35)\]
    
    \newpage
    \noindent答案:
    \[\text{1、} -\frac54\text{(提示:带分数可以把整数部分和分数部分分开)}\]
    \[\text{2、} -1\frac13\]
    \[\text{3、} -13\]
    \[\text{4、} -\frac78\]
    \[\text{5、} 1\]
    \[\text{6、} 2\frac17\]
    \[\text{7、} -11\]
    \[\text{8、} -\frac5{14}\]
    \[\text{9、} -3311.5\]
    \[\text{10、} 20\]

\end{document}
