% fleqn,控制公式左对齐
\documentclass[fleqn]{article}
\usepackage{xeCJK}

% mathtools数学环境
\usepackage{mathtools}
% 控制页边距
\usepackage[a4paper,left=2cm,right=3cm,top=2cm,bottom=3cm]{geometry}
\setCJKmainfont{SimSun}

% title
\title{重难点演练:有理数的四则运算2}
% 不显示时间
%\date{}
% 以上基本不用修改

% 自定义页眉
\usepackage{fancyhdr}
\pagestyle{fancy}
\chead{先算乘方,后算乘除,最后算加减,有括号的先算括号里的}
\cfoot{学而思暑期班\phantom{how}第3讲随材}
\begin{document}
  % 这里设置标题的样式
  \hspace{4cm}{\Large 重难点演练3:有理数的混合运算}
  \\
	\noindent\[\text{1、}(-4)^2-5^2\times(-\frac25)\]
	\[\text{2、}(-6)^2\div(-4)+(-2^2)\times2\]

	\[\text{3、}3^4\times\frac1{27}+(-2)^2\times\frac12\div(-2)\]

	\[\text{4、}(-\frac12)-2\times0.5^2+3^2\div(-3)\]

	\[\text{5、}-3^2-(-3)^2\times(-\frac13)+(-3)^3\div3\]

	\[\text{6、}-72-3\times\{3\times[(-2)+(-1)^{2006}]\div(\frac56-\frac23)\}\]

	\[\text{7、}(-0.125)^7\times8^9\]

	\[\text{8、}3^2\times(-\frac13)^3-2^4\div(-\frac12)\]

	\[\text{9、}-2^4-(-2)^4+(-1)^{2005}-(-1)^{2016}\]

	\[\text{10、}(\frac23)^2\times(-3)^2-\frac{2^2}3\div\frac13\]

	\[\text{11、}-|8\times(-2)^3|-[(-\frac12)^4\times16]^3\]

	\[\text{12、}[1-(1-0.5\times\frac13)]\times[2-(-3)^2]\]

	\[\text{13、}(-2)^3\times(-1)^2-|-12|\div(\frac34-1)\]

	\[\text{14、}(-2)^3-3\times[(-4)^2+2]-(-3)^2\div(-2)\]

	\[\text{15、}-1^{2016}-(1-0.5)\times\frac13\times[3-(-3)^2]\]

	\[\text{16、}(\frac35-\frac12-\frac7{12})\times(60\times\frac37-60\times\frac17+60\times\frac57)\]

	\newpage
	\noindent答案:\\
	\[\text{1、}26\]
	\[\text{2、}-17\]
	\[\text{3、}2\]
	\[\text{4、}-4\]
	\[\text{5、}-15\]
	\[\text{6、}-18\]
	\[\text{7、}-64\]
	\[\text{8、}31\frac23\]
	\[\text{9、}-34\]
	\[\text{10、}0\]
	\[\text{11、}-65\]
	\[\text{12、}-\frac76\]
	\[\text{13、}40\]
	\[\text{14、}-57\frac12\]
	\[\text{15、}0\]
	\[\text{16、}-29\]
\end{document}
