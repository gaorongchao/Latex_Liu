\documentclass[fleqn,UTF8]{ctexart}                                       % fleqn,控制公式左对齐

\usepackage[a4paper,left=2cm,right=3cm,top=2cm,bottom=3cm]{geometry} % 控制页边距

\usepackage{mathtools}                                               % mathtools数学环境
\usepackage{xeCJK}
\usepackage{fancyhdr}                                                % 自定义页眉
\pagestyle{fancy}
\setCJKmainfont{SimSun}
\chead{世界不灭,挑战不止!\phantom{how}\thepage}                                                             % 页眉
\cfoot{学而思\phantom{how}第5讲随材}                                                % 页脚

\title{}                                              % title
\date{}                                                              % 不显示时间


\begin{document}

\hspace{5cm}{\Large \kaishu 整式的加减}                                 % 这里设置标题的样式

\noindent1、\kaishu \textbf{同类项:}所含字母相同,并且相同字母的指数也相同的单项式叫做同类项。\\
2、\textbf{合并同类项:}把同类项的系数相加,所含字母和字母的指数不变。\\
\phantom{2、\textbf{合并同类项:}} 找同类项 => 在一起 => 系数相加减。\\
3、\textbf{去括号:}“$+$”:去掉括号以后,各项符号都不改变。\\
\phantom{3、去括号:}“$-$”:去掉括号以后,各项的符号都改变。\\
\phantom{3、去括号:} 括号前面的数要和括号里的每一项相乘\\
\phantom{3、去括号:=} $-3(-a+2b)$ (两步走)\\
\phantom{3、去括号:} $ =-(-3a+6b)$(3先和括号里的每一项相乘)\\
\phantom{3、去括号:} $ =3a-6b$(去括号)\\
4、\textbf{添括号:}“$+$”:添上括号以后,各项符号都不改变。\\
\phantom{3、去括号:}“$-$”:添上括号以后,各项的符号都改变。\\
5、\textbf{整式的加减:}去括号+合并同类型\\
6、\textbf{做差法比较大小:}$A-B>0,A>B$\\
7、\textbf{与某一项无关:}比如,不含ab项,说明所有ab项的系数和为0。\\
\newpage

\hspace{5cm}{\Large \kaishu 规范解题4例}\\
例5:(3)、先化简,再求值:\\
\[\text{已知:}(a+2)^2+|b-\frac14|=0,\text{求}5a^2b-[2a^2b-(ab^2-2a^2b)-4]-2ab^2\text{的值}\]
\[\text{解:因为}(a+2)^2+|b-\frac14|=0\]
\[\text{\phantom{解:}所以}a+2=0,b-\frac14=0\]
\[\text{\phantom{解:}所以}a=-2,b=\frac14\]
\[\text{\phantom{解:}原式}=5a^2b-2a^2b+(ab^2-2a^2b)+4-2ab^2\text{(去大括号)}\]
\[\text{\phantom{解:原式}}=5a^2b-2a^2b+ab^2-2a^2b+4-2ab^2\text{(去小括号)}\]
\[\text{\phantom{解:原式}}=5a^2b-2a^2b-2a^2b+ab^2-2ab^2+4\text{(找同类项,在一起)}\]
\[\text{\phantom{解:原式}}=a^2b-ab^2+4\text{(合并同类项)}\]\\
\[\text{\phantom{解:原式}}\text{把}a=-2,b=\frac14\text{代入,得:}\]
\[\text{\phantom{解:原式}}(-2)^2\times\frac14-(-2)\times(\frac14)^2+4\]
\[\text{\phantom{解:原}}=1+\frac18+4\]
\[\text{\phantom{解:原}}=5\frac18\]\\
\[\text{例6(1)、若}a+b=4,ab=-2,\text{求代数式}(4a-3b-2ab)-(a-6b+ab)\text{的值}\]
\[\text{解:原式}=4a-3b-2ab-a+6b-ab\text{(去括号)}\]
\[\text{\phantom{解:原式}}=4a-a-3b+6b-2ab-ab\text{(找同类项,在一起)}\]
\[\text{\phantom{解:原式}}=3a+3b-3ab\text{(找同类项,在一起)}\]\\
\[\text{\phantom{解:}把}a+b=4,ab=2\text{代入}\]
\[\text{\phantom{解:=}}3a+3b-3ab\]
\[\text{\phantom{解:}}=3(a+b)-3ab\]
\[\text{\phantom{解:}}=3\times4-3\times(-2)\]
\[\text{\phantom{解:}}=12+6\]
\[\text{\phantom{解:}}=18\]\\
\newpage
\[\text{例6(2)、若}2x^2-3x+1=0,\text{求代数式}5x^2-[5x^2-2(2x^2-x)+4x-5]\text{的值}\]
\[\text{解:原式}=5x^2-5x^2+2(2x^2-x)-4x+5\text{(去中括号)}\]
\[\text{\phantom{解:原式}}=(4x^2-2x)-4x+5\text{(把2乘进括号)}\]
\[\text{\phantom{解:原式}}=4x^2-2x-4x+5\text{(去小括号)}\]
\[\text{\phantom{解:原式}}=4x^2-6x+5\text{(合并同类项)}\]\\
\[\text{\phantom{解:}因为}2x^2-3x+1=0\]
\[\text{\phantom{解:}所以}2x^2-3x=-1\text{(引导孩子观察题目,给出的条件和要求的内容有什么联系)}\]\\
\[\text{\phantom{解:=}}4x^2-6x+5\]
\[\text{\phantom{解:}}=2(2x^2-3x)+5\]
\[\text{\phantom{解:}把}2x^2-3x=-1\text{代入,得}\]
\[\text{\phantom{解:=}}2\times(-1)+5\]
\[\text{\phantom{解:}}=(-2)+5\]
\[\text{\phantom{解:}}=3\]\\
\[\text{例7、若}A=2y^2+3ky-2y-1,B=-y^2+ky-1\text{且}3A+6B\text{的值与}y\text{无关,求k的值}\]
\[\text{解:\phantom{=}}3A+6B\]
\[\text{\phantom{解:}}=3(2y^2+3ky-2y-1)+6(-y^2+ky-1)\]
\[\text{\phantom{解:}}=6y^2+9ky-6y-3-6y^2+6ky-6\text{(去括号)}\]
\[\text{\phantom{解:}}=6y^2-6y^2+9ky-6y+6ky-3-6\text{(找同类项,在一起)}\]
\[\text{\phantom{解:}}=12y^2+(9k-6+6k)y-9\text{(找同类项,在一起)}\]
\[\text{\phantom{解:}}=12y^2+(15k-6)y-9\]\\
\[\text{\phantom{解:}}\text{因为与y项无关}\]
\[\text{\phantom{解:}}\text{所以}(15k-6)y=0\]
\[\text{\phantom{解:}}\text{所以}15k-6=0\]
\[\text{\phantom{解:}}\text{所以}k=\frac25\]
\end{document}
