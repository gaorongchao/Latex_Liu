% fleqn,控制公式左对齐
\documentclass[fleqn]{article}
\usepackage{xeCJK}
\usepackage{listings}

% mathtools数学环境
\usepackage{mathtools}
% 控制页边距
\usepackage[a4paper,left=3cm,right=3cm,top=2cm,bottom=3cm]{geometry}
\setCJKmainfont{SimSun}

% title
\title{重难点演练:有理数的四则运算2}
% 不显示时间
%\date{}
% 以上基本不用修改

% 自定义页眉
\usepackage{fancyhdr}
\pagestyle{fancy}
\chead{$a\times10^n,1\leq |a| < 10$}
\cfoot{学而思暑期班\phantom{how}第3讲随材}
\begin{document}
  % 这里设置标题的样式
  \hspace{4cm}{\Large 挑战中考之科学记数法}
	\\
	\\

	\noindent科学记数法做题步骤:\\
	1、确定$a\times10^n$中a的值,$1\leq|a|<10$\\
	2、确定n的值,根据小数点移动的位数。\\
	\\
	\\
	\noindent1、(2013南京第10题)第二节亚洲青年运动会将于2013年8月16日至24日在南京举办,在此期间约有13000名青少年志愿者提供服务,将13000用科学计数法表示为\_。\\
	\\
	2、(2013盐城第11题)2013年4月20日,四川省雅安市芦山县发生7.0级地震.我市爱心人士情系灾区,积极捐款,截止到5月6日,市红十字会共收到捐款约1400000元,这个数据用科学记数法可表示为\_。\\
	\\
	3、(2013连云港第4题)为了传承和弘扬港口文化,我市将投入6000万元建设一座港口博物馆,其中6000万用科学记数法可以表示为\_。\\
	\\
	4、(2013苏州第5题)世界文化遗产长城总长约为6700000m,若将6700000用科学记数法表示为$6.7×10^n$(n是正整数),则n的值为\_。\\
	\\
	5、(2013泰州第9题) 2013年第一季度,泰州市共完成工业投资22300000000元,22300000 000这个数可用科学记数法表示为\_。\\
	\\
	6、(2013无锡第12题)去年,中央财政安排资金 8200000000 元,免除城市义务教育学生学杂费,支持进城务工人员随迁子女公平接受义务教育,这个数据用科学记数法可表示为\_元.\\
	\\
	7、(2013徐州第3题)2013年我市财政计划安排社会保障和公共卫生等支出约1820000000元支持民生幸福工程,该数据用科学记数法表示为\_。\\
	\\
	8、(2013扬州第9题)据了解,截止2013年5月8日,扬泰机场开通一年,客流量累计达到450000人次.数据450000用科学记数法可表示为\_。\\
	\\
	9、(2013南京第一题)计算$12-7\times(-4)+8\div(-2)$的结果是\_。\\
	\\
	思考题:\\
	\[\text{1、}1+\frac12+\frac1{2^2}+\frac1{2^3}+\frac1{2^4}+\frac1{2^5}+\frac1{2^6}\]
	\\
	\newpage
	\noindent答案:\\
	1、$1.3\times10^4$\\
	2、$1.4\times10^6$\\
	3、$6\times10^7$\\
	4、$6$\\
	5、$2.23\times10^{10}$\\
	6、$8.2\times10^9$\\
	7、$1.82\times10^9$\\
	8、$4.5\times10^5$\\
	9、$36$\\
	\noindent\[\text{思考题:}2-\frac1{2^6}\]
\end{document}
