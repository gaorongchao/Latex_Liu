% fleqn,控制公式左对齐
\documentclass[fleqn]{article}
\usepackage{xeCJK}

% 自定义页眉
\usepackage{fancyhdr}
\pagestyle{fancy}
\chead{减去一个数等于加上这个数的相反数}
\cfoot{学而思暑期班\phantom{how}第二讲随材}
% mathtools数学环境
\usepackage{mathtools}
% 控制页边距
\usepackage[a4paper,left=2cm,right=3cm,top=2cm,bottom=3cm]{geometry}
\setCJKmainfont{SimSun}

% title
\title{重难点演练:有理数的四则运算1}
% 不显示时间
\date{}
\begin{document}
  % 这里设置标题的样式
  \hspace{4cm}{\Large 重难点演练:有理数的四则运算1}
  \\
  \noindent加减混合运算5题:\\
  加减法则:\\
  同号两数相加,取相同的符号,并把绝对值相加。\\
  异号两数相加,取绝对值较大的符号,并用较大的绝对值减去较小的绝对值\\
  减法法则:\\
  减去一个数等于加上这个数的相反数。\\
  所有的减法首先转化成加法,然后计算\\
	\[\text{1、}24-(-6)+(-25)-15\]

	\[\text{2、}12-(-18)+(-7)-15\]

	\[\text{3、}-6-8-2+3.54-4.72+16.46-5.28\]

	\[\text{4、}15-(+5\frac 56)-(+3\frac 37)+(-2\frac 16)-(+6\frac 47)\]

	\[\text{5、}(-5\frac 23)-(+\frac 14)+(-\frac 13)+(+5)-(-1\frac 14)\]

  \noindent乘除混合运算5题:\\
  乘法法则:同号得正,异号得负,奇负偶正\\
  除法法则:除以一个数等于乘以这个数的倒数\\
  所有的除法先转换成乘法,然后再计算\\
	\[\text{1、}1.25\times\frac59\times(-4)\times0.6\]

	\[\text{2、}|-16|\times(-\frac34-\frac12+\frac38)\]

	\[\text{3、}(-\frac12-\frac34+\frac56-0.75)\div(-\frac1{24})\]

	\[\text{4、}11\frac13\times(-\frac2{17})\times18\frac23\div(-\frac79)\]

	\[\text{5、}12\div(-3-\frac14+1\frac13)\]

  \newpage
  \noindent 答案:\\
  加减法:\\
	\[\text{1、}-10\]
	\[\text{2、}8\text{减法先化成加法}\]
	\[\text{3、}-6\text{减法化加法,然后小数部分凑整}\]
	\[\text{4、}-3\text{减法化成加法后,同分母放一起计算}\]
	\[\text{5、}0\]
  乘除法:\\
	\[\text{1、}-\frac35\]
	\[\text{2、}-14\]
	\[\text{3、}28\]
	\[\text{4、}32\]
	\[\text{5、}-\frac{144}{23}\]
  \end{document}
